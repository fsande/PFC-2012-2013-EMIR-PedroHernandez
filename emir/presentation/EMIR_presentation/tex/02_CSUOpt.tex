%\section{Aplicación}
%%%%%%%%%%%%%%%%%%%%%%%%%%%%%%%%%%%%%%%%%%%%%%%%%%%%%%%%%%%%%%
\begin{frame}[fragile]
    \frametitle{\CSUO{}}
    \block{Entrada/Salida}
    \endblock{}
\begin{lstlisting}[numbers=none]
./csuoptimizer [opciones] entrada.xml
\end{lstlisting}

\begin{lstlisting}[linewidth=\linewidth,numbers=none,language=XML,basicstyle=\ttfamily\scriptsize]
<Observables [widht=``Ancho_cielo'' height=``Alto_cielo'']>
    <Element id=``Identificador''>
        <X>PosX</X>
        <Y>PosY</Y>
        <Prioridad>[1..99]</Prioridad>
    </Element>
<!-- ...  -->
</Observables>
\end{lstlisting}
\end{frame}
%%%%%%%%%%%%%%%%%%%%%%%%%%%%%%%%%%%%%%%%%%%%%%%%%%%%%%%%%%%%%%

%%%%%%%%%%%%%%%%%%%% Code  %%%%%%%%%%%%%%%%%%%%%%%%%%%%%%%%%%%
\begin{frame}[fragile]
    \frametitle{\CSUO{}}
    \block{Entrada/Salida}
    \endblock{}
    \begin{lstlisting}[linewidth=\linewidth,numbers=none,language=XML,basicstyle=\ttfamily\footnotesize]
<Apuntado>
    <X>ValorX</X>
    <Y>ValorY</Y>
	  <PA>AnguloPosicion</PA>
	  <Configuracion>
		    <Barra>Posicion1</Barra>
		    <Barra>Posicion2</Barra>
            <!-- ...  -->
		    <Barra>Posicion54</Barra>
		    <Barra>Posicion55</Barra>
	  </Configuracion>
</Apuntado>
\end{lstlisting}
\end{frame}
%%%%%%%%%%%%%%%%%%%%%%%%%%%%%%%%%%%%%%%%%%%%%%%%%%%%%%%%%%%%%%

%%%%%%%%%%%%%%%%%%%% Code  %%%%%%%%%%%%%%%%%%%%%%%%%%%%%%%%%%%
%\begin{frame}[fragile]
%\frametitle{}
%\block{}
%\begin{lstlisting}[linewidth=\textwidth, numbers=none,basicstyle=\ttfamily\scriptsize]
%$ ./csuoptimizer --help
%
%Pointing optimizer program for Emir
%
%Usage: ./csuoptimizer [options] <input file>"
%
%Available options:
%--beams: activates beam switching (default is off).
%--dat: Changes output format to .dat (default is XML).
%--dbscan: Activates the use of DBScan cluster detection algorithm (default is off).
%--grasp: Uses the GRASP heuristic to enhance the results (default is off).
%--graphic: Activates graphic step by step execution for debugging purposes. It slows execution (default is off).
%--noborder: Avoids objects in the border of the bars.
%-NR X: X stands for the number of CSU rotations to perform when searching. Default is 20.
%--verbose: Generates the 'verbose.txt' text file containing information related to the program execution.
%
%The <input file> must be a XML file with this format:
%
%<Observables [widht=Ancho_cielo height=Alto_cielo]>
%		<Element id=\"Identificador\">
%				<x>PosX</x>
%				<y>PosY</y>
%				<Prioridad>[1..99]</Prioridad>
%		</Element>
% 				 ···
%</Observables>
%\end{lstlisting}
%\endblock{}
%\end{frame}
%%%%%%%%%%%%%%%%%%%%%%%%%%%%%%%%%%%%%%%%%%%%%%%%%%%%%%%%%%%%%%%

%%%%%%%%%%%%%%%%%%%%%%%%%%%%%%%%%%%%%%%%%%%%%%%%%%%%%%%%%%%%%%
\begin{frame}
    \frametitle{\CSUO{}}
    \block{Opciones}
    \begin{itemize}[<+->]
    \item Uso del DBScan (No)
    \item Elegir número de rotaciones (20)
    \item Elegir tipo de salida (XML)
    \item Tipo de barra (Normal)
    \end{itemize}
    \endblock{}
\end{frame}
%%%%%%%%%%%%%%%%%%%%%%%%%%%%%%%%%%%%%%%%%%%%%%%%%%%%%%%%%%%%%%

%%%%%%%%%%%%%%%%%%%%%%%%%%%%%%%%%%%%%%%%%%%%%%%%%%%%%%%%%%%%%%
\begin{frame}
    \frametitle{Tipos de barras}
    \block{Barras normales}
    \endblock{}
    \includegraphics[width=0.4\linewidth]{FIGURES/BarraPunto1}
    \hspace{5mm}
    \includegraphics[width=0.4\linewidth]{FIGURES/Bueno1}
\end{frame}
%%%%%%%%%%%%%%%%%%%%%%%%%%%%%%%%%%%%%%%%%%%%%%%%%%%%%%%%%%%%%%

%%%%%%%%%%%%%%%%%%%%%%%%%%%%%%%%%%%%%%%%%%%%%%%%%%%%%%%%%%%%%%
\begin{frame}
    \frametitle{Tipos de barras}
    \block{Barras sin bordes}
    \endblock{}
    \includegraphics[width=0.4\linewidth]{FIGURES/BarraPunto2}
    \hspace{5mm}
    \includegraphics[width=0.4\linewidth]{FIGURES/Bueno2}
\end{frame}
%%%%%%%%%%%%%%%%%%%%%%%%%%%%%%%%%%%%%%%%%%%%%%%%%%%%%%%%%%%%%%

%%%%%%%%%%%%%%%%%%%%%%%%%%%%%%%%%%%%%%%%%%%%%%%%%%%%%%%%%%%%%%
\begin{frame}
    \frametitle{Tipos de barras}
    \block{Barras con Beam Switching}
    \endblock{}
    \includegraphics[width=0.4\linewidth]{FIGURES/BarraPunto3}
    \hspace{5mm}
    \includegraphics[width=0.4\linewidth]{FIGURES/Bueno3}
\end{frame}
%%%%%%%%%%%%%%%%%%%%%%%%%%%%%%%%%%%%%%%%%%%%%%%%%%%%%%%%%%%%%%

%%%%%%%%%%%%%%%%%%%%%%%%%%%%%%%%%%%%%%%%%%%%%%%%%%%%%%%%%%%%%%
\begin{frame}
    \frametitle{Tipos de barras}
    \block{Barras con ambas opciones}
    \endblock{}
    \includegraphics[width=0.4\linewidth]{FIGURES/BarraPunto4}
    \hspace{5mm}
    \includegraphics[width=0.4\linewidth]{FIGURES/Bueno4}
\end{frame}
%%%%%%%%%%%%%%%%%%%%%%%%%%%%%%%%%%%%%%%%%%%%%%%%%%%%%%%%%%%%%%
