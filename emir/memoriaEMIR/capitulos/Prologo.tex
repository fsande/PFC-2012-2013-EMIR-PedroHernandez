%
% ---------------------------------------------------
%
% Proyecto Final de Carrera: EMIR
% Autor: Pedro Hernández Martín <alu3679@etsii.ull.es>
% Capítulo: Introduccion
% Fichero: Prologo.tex
%
% ----------------------------------------------------
%
\chapter*{Prólogo} \label{chap:estado}
En este documento se refleja el trabajo realizado durante el tiempo que ha
llevado completar este Proyecto de Final de Carrera (PFC) de los estudios en
Ingeniería en Informática cursados en la Escuela Técnica Superior de Ingeniería
Informática (ETSII) de la Universidad de La Laguna (ULL). El Proyecto Final
de Carrera se ha desarrollado en torno a un elemento del proyecto de investigacion
EMIR y está destinado a solventar un problema real con el que se encuentran los
investigadores a la hora de utilizar de manera óptima los recursos de los que
disponen.

En los capítulos que componen esta memoria se intenta sintetizar el trabajo
realizado, mediante la aplicación de los conceptos aprendidos y la práctica
adquirida durante toda la carrera en la ETSII de la ULL. El objetivo de este
documento no trata sólo de recoger la memoria del PFC sino de
servir, a su vez, como manual de usuario de la aplicación desarrollada.

El Espectrógrafo Multiobjeto InfraRojo (EMIR) \cite{Web:EMIR}, es un instrumento
de apoyo para el proyecto GOYA \cite{Web:GOYA} (\textit{Galaxias: Orígenes y Evolución a Alto z}) desarrollado por el Instituto
de Astrofísica de Canarias (IAC) \cite{Web:IAC}. GOYA es un programa
científico e instrumental diseñado para el Gran Telescopio de Canarias (GTC)
\cite{Web:GTC}, liderado por el Instituto de Astrofísica de Canarias (IAC) y
situado en La Palma. El principal objetivo científico es caracterizar la
población de galaxias durante la época de máxima formación estelar en la
historia del universo. Según estudios recientes, esta época crítica ocurrió
cuando el universo tenía un 10-40\% de su edad actual, lo que corresponde a
desplazamientos al rojo 1 $<$ z $<$ 2. A estos desplazamientos al rojo, la
ventana óptica - la región del espectro que ha sido estudiada en mayor
profundidad en las galaxias cercanas - está desplazada hacia el infrarrojo
cercano (1-2.5 micras). Los principales estudios espectroscópicos por debajo de
1.8 micras están siendo planeados actualmente para investigar las propiedades
ópticas de las galaxias en el sistema de reposo y la formación estelar global
del universo hasta z=1. Se propone llevar a cabo un estudio comprensivo de las
galaxias con 2 $<$ z $<$ 3, que incluya: morfología, estructura, cinemática,
población estelar, tasa de formación estelar, metalicidad, luminosidad y
funciones de masa, agrupamiento en cúmulos, y estructura a gran escala. El
objetivo es entender la naturaleza de estas galaxias distantes y evaluar su
papel en la historia de la formación estelar del universo, comparando
directamente sus propiedades ópticas en el sistema de reposo con las de la
población cercana. GOYA será el primer estudio importante que extenderá estas
investigaciones al universo a alto z.

EMIR es un cámara de gran campo y espectrógrafo multiobjeto de resolución
intermedia en el infrarrojo cercano para el telescopio GTC. Está equipado, entre
otros, con tres subsistemas de alta tecnología de última generación, algunos
especialmente diseñados para este Proyecto: un sistema robótico reconfigurable
de rendijas (para obtener espectros de en torno a 50 objetos simultáneamente);
elementos dispersores formados mediante la combinación de redes de difracción de
alta calidad, fabricadas mediante procedimientos fotorresistivos, y prismas
convencionales de gran tamaño, y el detector HAWAII-2 de Rockwell, diseñado para
el infrarrojo cercano con un formato de 2048$\times$2048 píxeles, y dotado de un
novedoso sistema de control, desarrollado por el equipo del proyecto. EMIR es un
instrumento de segunda generación que se instalará en el foco Nasmyth de GTC.

El objetivo fundamental de nuestro Proyecto ha consistido en el desarrollo de
una aplicación, completamente funcional, que resuelva de manera óptima la
configuración del subsistema de rendijas mencionando anteriormente.

Esta Memoria del PFC está estructurada en torno a siete capítulos, cuyos
contenidos se describen brevemente a continuación.

El primer capítulo, Motivaciones, se pretende situar al lector en el contexto
del proyecto EMIR, explicando detalladamente sus objetivos, componentes,
funcionamiento y tendencias futuras. También se definirá el problema que nos
atañe, y se repasará la importancia de este proyecto a nivel internacional.

En el segundo capítulo se definen los objetivos marcados a la hora de realizar
este Proyecto, de los cuales se ha conseguido completar satisfactoriamente la
gran mayoría.

El capítulo Tecnologías y herramientas relacionadas hace un recorrido por
aquellas tecnologías que se encuentran a nuestro alcance y que, si bien no han
sido seleccionadas para formar parte del Proyecto, han sido investigadas con
esta finalidad. Se intenta hacer comprender las decisiones tomadas durante el
transcurso del Proyecto para elegir qué tecnologías se utilizaban o cuáles
fueron los motivos por los que se desecharon.

El cuarto capítulo, Algoritmos, explica los pasos que realiza la aplicación para
obtener el resultado, así como el por qué de la elección de estos métodos. Entre
su contenido se encuentran pseudo-código y esquemas que tratan de aportar aún más
información sobre su comportamiento.

En el capítulo referente a la Aplicación, se halla el manual de usuario y la
documentación del código entregado. Aquí se especificarán los requisitos previos
necesarios para instalar la aplicación, así como la explicación
de cómo instalarla; también se hará un recorrido por el contenido de los
directorios. En la documentación se encuentra la descripción de las clases,
métodos y estructuras más relevantes de nuestra aplicación, para un mayor
entendimiento de la misma.

El sexto capítulo recopila los resultados, comprendidos por una breve
descripción del ejemplo introducido como entrada, capturas de la salida de
nuestra aplicación, una tabla de tiempos y resultados obtenidos variando los
parámetros de entrada y una explicación de estos tiempos.

Se finaliza con unas conclusiones, que recogen las impresiones sobre el trabajo
realizado y aportan, a su vez, una futura línea en la que se puede seguir
desarrollando el Proyecto.
