%
% ---------------------------------------------------
%
% Proyecto Final de Carrera: EMIR
% Autor: Pedro Hernández Martín <alu3679@etsii.ull.es>
% Capítulo: Detalles del problema
% Fichero: Cap_conclusiones.tex
%
% ----------------------------------------------------
%

\chapter{Conclusiones y Trabajos Futuros} \label{chap:conclusions}
En este PFC, se ha abordado la resolución del problema de minimización del número de apuntados
del telescopio GTC necesarios para observar mediante el instumento EMIR un conjunto de objetos
de interés astronómico.
Se trata de un problema real que tienen los investigadores del proyecto EMIR y ello ha supuesto para
el autor un acicate en nuestro esfuerzo para acreditar que somos capaces de resolver eficientemente
problemas de ingeniería.

El resultado más tangible del PFC es la aplicación \CSUO{}. 
Se trata de una pequeña aplicación escrita en C++ siguiendo el paradigma orientado a objetos de unas
3300 líneas de código.
No obstante quisiéramos destacar desde el punto de vista académico que el PFC, como no puede ser de otro modo, 
ha representado un escenario real de trabajo en el que el autor ha
podido desplegar de forma práctica técnicas, metodologías, herramientas y conceptos estudiados en los años de discencia en la ETSI Informática.

Las mayores dificultades halladas en el desarrollo del proyecto han sido:
\begin{itemize}
\item El estudio y selección de algoritmos que resuelvan el problema planteado.
\item La comprensión precisa de las características intrínsecas al problema.
\item La adaptación del lenguaje de la astronomía al campo que nos es propio de la computación.
\item La mejora de la calidad de las soluciones obtenidas por la aplicación.
\item La reducción de los tiempos de ejecución del algoritmo que finalmente se ha implementado.
\end{itemize}
Queremos destacar la importancia del trabajo de depuración y optimización del código que
compone el \CSUO{} para obtener una versión que resuelve satisfactoriamente el problema propuesto.
Asimismo es destacable el trabajo realizado por el autor para lograr elaborar un producto listo para usar 
por parte de usuarios ajenos a la informática, poniendo a su disposición esta memoria del PFC, que constituye
al mismo tiempo la mejor documentación disponible para la propia aplicación desarrollada.
Importante ha sido en el desarrollo el trabajo con las
herramientas de versionado y gestión de repositorio, sin las cuales las tareas de
depuración y publicación de \textit{release} hubieran sido mucho más tediosas.

En este documento se han revisado los logros conseguidos en la ejecución del proyecto. 
Relacionándolos directamente con los objetivos planteados inicialmente, podemos
señalar como más significativos los siguientes logros:

\begin{itemize}
\item Se ha realizado una investigación sobre posibles métodos para resolver el problema propuesto.
\item Se ha desarrollado una aplicación plenamente operativa que resuelve el problema de optimización de
      los apuntados de forma que consideramos satisfactoria.
\item El software que se entrega como resultado ha sido desarrollado teniendo en mente que posiblemente sea
      tomado como punto de partida para desarrollos ulteriores que, utilizando esta implementación como base,
			la doten de funcionalidades adicionales, particularmente en el área de interfaz de usuario.
			Este punto de vista ha obligado al autor a un trabajo concienzudo de documentación y estructuración
			razonada del código.
\item Se aporta el presente documento que constituye la documentación técnica de la aplicación desarrollada (Capítulo \ref{chap:aplication}).
      Este documento será de gran ayuda para cualquier ingeniero que tenga que hacerse cargo posterior de este desarrollo
			tanto a efectos de mejora, depuración u optimización.
\end{itemize}

Consideramos por otra parte que el problema propuesto por el equipo investigador del proyecto EMIR no queda cerrado con la
conclusión de este PFC.
Debido a las limitaciones temporales que afectan a un PFC en Informática hay diversos aspectos del trabajo realizado que
claramente cabe mejorar.
Algunas mejoras que hubieran tenido cabida en el marco del PFC no entrañan gran dificultad. 
Si no han sido desarrolladas, ello se debe a que hasta el último momento, los esfuerzos del trabajo se han centrado exclusivamente
en la mejora de las soluciones encontradas por la aplicación.
Reseñamos a continuación en forma de posibles trabajos futuros a acometer en el marco del Proyecto EMIR algunas de estas mejoras:
\begin{itemize}
\item Un estudio más concienzudo de la calidad de las soluciones que se obtienen con el \CSUO{} respecto a las soluciones óptimas
      ha quedado fuera del trabajo realizado.
\item Hay margen de mejora en cuanto a los tiempos de respuesta del \CSUO{}. 
      Aunque se encuentran dentro de los márgenes inicialmente establecidos en la especificación de requisitos, pensamos que no es un aspecto
			difícil de mejorar.
			Un análisis preliminar mediante profiling de la aplicación nos hace ser conocedores de las funciones que habría que optimizar y
			una tecnología de paralelización utilizando OpenMP se nos ocurre como la solución a implementar inicialmente.
\item La interfaz de usuario es otro aspecto que claramente merece atención en cuanto a necesidades de desarrollo.
      A pesar de que los usuarios del \CSUO{} pertenecen al ámbito científico-tecnológico, en el que hay una mayor permisividad respecto
			a las limitaciones en cuanto a interfaz de usuario de una aplicación, pensamos que estos usuarios se beneficiarían de una
			pasarela que permitiera una selección cómoda de los objetos a observar.
			En este sentido, hemos optado por elegir XML como formato de entrada de datos a la aplicación, lo cual flexibiliza la conexión con una
			futura GUI.
\end{itemize}
