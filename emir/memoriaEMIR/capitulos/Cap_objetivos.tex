%
% ---------------------------------------------------
%
% Proyecto Final de Carrera: EMIR
% Autor: Pedro Hernández Martín <alu3679@etsii.ull.es>
% Capítulo: Estado del arte
% Fichero: Cap2_estado_del_arte.tex
%
% ----------------------------------------------------
%

\chapter{Objetivos} \label{chap:objetivos}

A continuación se listan los objetivos propuestos para este PFC,
de los cuales se han completado totalmente la mayoría mientras que
la dedicación a algunos de ellos se ha tenido que minimizar u obviar por falta de tiempo.

\begin{itemize}
\item Desarrollar una aplicación plenamente funcional que resuelva de modo
óptimo el problema de optimización de apuntados de la CSU del proyecto EMIR.
\item Realizar la resolución del problema en el menor tiempo posible.
\item Tratar las prioridades entre elementos observables.
\item Contemplar el uso por parte de los investigadores del método conocido como \textit{beam switching}.
\item Estudiar diferentes alternativas para la resolución del problema
computacional planteado, así como realizar una revisión bibliográfica de métodos
y técnicas de optimización relacionados con dicho problema.
\item Involucrarse en el conocimiento de un proyecto real de investigación
aplicada, el proyecto EMIR, siendo capaz de aportar al mismo un elemento
importante para su desarrollo.
\item Conocer con cierto grado de detalle la tecnología involucrada en el
desarrollo del proyecto EMIR así como el contexto en el que esta información se
utiliza.
\item Participación en un proceso de ``release'' de software (empaquetado,
testing, validación, etc).
\item Poner en práctica de modo efectivo en una aplicación real los
conocimientos adquiridos en la titulación en materia de Programación e
Ingeniería del Software.
\item Diseño e implementación de una interfaz gráfica amigable al usuario. Éste
punto ha sido eliminado por falta de tiempo, dejando como parte gráfica la
visualización de los resultados.
\end{itemize}
