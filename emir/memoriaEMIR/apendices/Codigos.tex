%
% ---------------------------------------------------
%
% Proyecto Final de Carrera: EMIR
% Autor: Pedro Hernández Martín <alu3679@etsii.ull.es>
% Capítulo: Añadidos a la memoria
% Fichero: Apendices.tex
%
% ----------------------------------------------------
%
\chapter{Códigos}
\section{\texttt{colisiones}} \label{app:colis}

\begin{lstlisting}[language=C++, numbers=none,basicstyle=\ttfamily\footnotesize,
                   caption={Código de la función que reduce las colisiones
                   aleatorias},]
set<CSU> colisiones (set<CSU> lista_ini, const int &total_puntos) {
  static int finalizar;
  finalizar = 0;
  static set<Element> puntos, p_parcial;
  static set<Element>::iterator p;
  sort(lista_ini, TAMPUN);
  static set<CSU>::iterator jt, it;
  static set<CSU> porcion, no_coli;
  set<CSU> lista_fin;
  no_coli.clear();
  while (finalizar != (lista_ini.size()+no_coli.size())) {
    lista_fin.clear();
    finalizar = lista_ini.size();
    set<Element> eliminados;
    eliminados.clear();
    it = lista_ini.begin();
    while (it != lista_ini.end()){
      static bool quitable;
      quitable = true;
      jt = lista_ini.begin();
      while (jt != lista_ini.end() && quitable){
        if (it != jt && (it->colision(*jt) || jt->colision(*it))) //Si colisiona, no lo podemos quitar
          quitable = false;
        jt++;
      }
      if (quitable) {
        lista_fin.insert(*it);
        p_parcial = it->getPoints();
        for (p = p_parcial.begin(); p != p_parcial.end(); p++) 
          eliminados.insert(*p);
        static set<CSU>::iterator borrar;
        borrar = it; //Guardamos la posicion que se va a elimimar
        it++; //pasamos a mirar la siguiente
        lista_ini.erase(borrar); //    
      }                                
      else                             
        it++;                          
    }                                  
    while (eliminados.size() != total_puntos) { //mientras queden puntos por mirar
      puntos.clear();
      p_parcial.clear();
      porcion.clear();
      it = lista_ini.begin(); //it es el principio de la lista de CSUs
		  // Cogemos los puntos del primero y los colocamos en la lista de mejora 
      p_parcial = it->getPoints(); //p_parcial son todos los puntos de la primera CSU
      for (p = p_parcial.begin(); p != p_parcial.end(); p++)
        puntos.insert(*p); //insertamos todos los puntos de la primera CSU en los totales
      porcion.insert(*it); //Ponemos esa CSU en el grupo que se va a mirar
      CSU ini = *it; //Almacenamos la primera CSU
      lista_ini.erase(it); //Borramos la CSU de la lista total
      jt = lista_ini.begin(); //jt es el principio de la lista de CSUs
      // Seleccionamos todos aquellos que colisionan con el anteriormente seleccionado
      while(!lista_ini.empty() && jt != lista_ini.end()) { //comprobamos cada uno con el primero de la lista
        if (ini.colision(*jt) || jt->colision(ini)) {
          porcion.insert(*jt); //Ponemos esa CSU en el grupo que se va a mirar
          p_parcial = jt->getPoints(); //p_parcial son los puntos de esa CSU
          for (p = p_parcial.begin(); p != p_parcial.end(); p++) 
            puntos.insert(*p); //Aniadimos los puntos al total
          static set<CSU>::iterator borrar;
          borrar = jt; //Guardamos la posicion que se va a elimimar
          jt++; //pasamos a mirar la siguiente
          lista_ini.erase(borrar); //
        }
        else {
          jt++;
        }
      }
      static bool continuar;
      continuar = true;
      // Si la colision no es necesaria teoricamente, se intenta reducir
      if (puntos.size()/NUM_BARRAS < porcion.size()) {
        static bool izq_menor;
        static CSU final;
        static set<CSU>::iterator fin;
        fin = porcion.end();
        fin--;
        final = *fin;
        izq_menor = (ini.gety() < final.gety())? true : false;
        set<CSU> nuevos = sacar_apuntados(puntos);
        sort(nuevos, TAMPUN);
        //Si se ha conseguido mejorar, volvemos a insertarlos para remirarlos
        if ((nuevos.size() < porcion.size()) || (nuevos.size() == porcion.size() && nuevos.begin()->size() > porcion.begin()->size())) {
          continuar = false;
          porcion = nuevos;
          for (it = porcion.begin(); it != porcion.end(); it++)
            lista_ini.insert(*it);
        }
      }
      // Si no se ha conseguido mejorar o la colision en necesaria, quitamos la primera CSU
      if (continuar) {
        it = porcion.begin();
        lista_fin.insert(*it);
        p_parcial = it->getPoints();
        for (p = p_parcial.begin(); p != p_parcial.end(); p++) {
          eliminados.insert(*p);
        }
        porcion.erase(it);
        for (it = porcion.begin(); it != porcion.end(); it++)
          lista_ini.insert(*it);
      }
    }
    lista_ini = lista_fin;
  }
  return lista_fin;
}

\end{lstlisting}

\section{\texttt{grasp}} \label{app:grasp}

\begin{lstlisting}[language=C++, numbers=none,basicstyle=\ttfamily\footnotesize,
                   caption={Método heurístico GRASP},]
set<CSU> grasp (set<CSU> resultado, set<Element> puntos) {
  int num_sol = 5;
  int tam_entrada = resultado.size();
  sort(resultado, TAMPUN);
  set<CSU> posibles, lista[num_sol], final;
  static set<CSU>::iterator it, jt, borrar;
  set<Element> p_finales;
  //PASO 1: generar N soluciones
  for (int i = 0; i < num_sol; i++)
    lista[i] = apuntadosRandom(puntos, i);
  //PASO 2: buscar mejor solucion
  for (int i = 0; i < num_sol; i++)
    if (lista[i].size() < resultado.size())
      resultado = lista[i];
  set<CSU> resultado_previo = resultado;
  //inicializar la solucion final
  it = resultado.begin();
  while (it != resultado.end()) {
      progreso();
    jt = it;
    jt ++;
    static bool col;
    col = false;
    while (jt != resultado.end() && !col) {
      progreso();
      if (it->colision(*jt) || jt->colision(*it))
        col = true;
      jt++;
    }
    if (!col) {
      final.insert(*it);
      static set<Element> contar;
      contar = it->getPoints();
      for (set<Element>::iterator cc = contar.begin(); cc != contar.end(); cc++)
        p_finales.insert(*cc);
      borrar = it;
      it++;
      resultado.erase(borrar);
    }
    else
      it++;
  }
  //PASO 3: Aniadir los restantes de S1 a la RCL
  for (it = resultado.begin(); it != resultado.end(); it++)
    posibles.insert(*it);
  for (int i = 0; i < num_sol; i++)
    for(it = lista[i].begin(); it != lista[i].end(); it++)
      posibles.insert(*it);
  //PASO 4: Seleccionar elementos de la RCL hasta que se cubran todos los puntos
  while (p_finales.size() != puntos.size()) {
    progreso();
    static int tam_point_max, limite;
    it = jt = posibles.begin();
    tam_point_max = it->size();
    limite = 0;
    it++;
    while (it->size() == tam_point_max) {
      it++;
      limite++;
    }
    if (limite != 0)
      limite = rand() % limite;
    for (int i = 0;  i < limite; i++)
      jt++;
    final.insert(*jt);
    static set<Element> contar;
    contar = it->getPoints();
    for (set<Element>::iterator cc = contar.begin(); cc != contar.end(); cc++)
      p_finales.insert(*cc);
    posibles.erase(jt);
  }}
  //PASO 5: Intentar mejorar el resultado obtenido.
  posibles.clear();
  posibles = final;
  posibles = colisiones(posibles, puntos.size());
  if (posibles.size() < resultado_previo.size()) {
    cout << "Mejorado con " << posibles.size() << endl;
    return posibles;
  }
  else {
    cout << "GRASP inutil" << endl;
    return resultado_previo;
  }
}

\end{lstlisting}

\section{Salida \texttt{exacto.xml}} \label{app:exacto.xml}

\begin{lstlisting}[language=XML, basicstyle=\ttfamily\footnotesize]
<Apuntado>
	<X>479.1</X>
	<Y>598.787</Y>
	<PA>0</PA>
	<Configuracion>
		<Barra>93.9</Barra>
		<Barra>115.9</Barra>
		<Barra>-21.1</Barra>
		<Barra>-70.1</Barra>
		<Barra>119.9</Barra>
		<Barra>-76.1</Barra>
		<Barra>94.9</Barra>
		<Barra>55.9</Barra>
		<Barra>-52.1</Barra>
		<Barra>-17.1</Barra>
		<Barra>65.9</Barra>
		<Barra>72.9</Barra>
		<Barra>-25.1</Barra>
		<Barra>-71.1</Barra>
		<Barra>89.9</Barra>
		<Barra>83.9</Barra>
		<Barra>98.9</Barra>
		<Barra>-23.1</Barra>
		<Barra>112.9</Barra>
		<Barra>-27.1</Barra>
		<Barra>7.9</Barra>
		<Barra>35.9</Barra>
		<Barra>-67.1</Barra>
		<Barra>29.9</Barra>
		<Barra>46.9</Barra>
		<Barra>109.9</Barra>
		<Barra>-42.1</Barra>
		<Barra>30.9</Barra>
		<Barra>-25.1</Barra>
		<Barra>-30.1</Barra>
		<Barra>-37.1</Barra>
		<Barra>-58.1</Barra>
		<Barra>-3.1</Barra>
		<Barra>6.9</Barra>
		<Barra>24.9</Barra>
		<Barra>8.9</Barra>
		<Barra>-59.1</Barra>
		<Barra>68.9</Barra>
		<Barra>75.9</Barra>
		<Barra>51.9</Barra>
		<Barra>-60.1</Barra>
		<Barra>11.9</Barra>
		<Barra>21.9</Barra>
		<Barra>-28.1</Barra>
		<Barra>102.9</Barra>
		<Barra>35.9</Barra>
		<Barra>107.9</Barra>
		<Barra>52.9</Barra>
		<Barra>-36.1</Barra>
		<Barra>-20.1</Barra>
		<Barra>112.9</Barra>
		<Barra>-9.1</Barra>
		<Barra>117.9</Barra>
		<Barra>86.9</Barra>
		<Barra>-70.1</Barra>
	</Configuracion>
</Apuntado>
\end{lstlisting}

